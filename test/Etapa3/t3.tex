\documentclass{gift}
\begin{document}
\begin{giftComentario}
  
\end  {giftComentario}
\giftNombre{Sobrecarga_P18_1}
\begin{giftFragmento}
Dada la siguiente definicion:\end  {giftFragmento}

\verb!foo a b c d e = (snd . fst) ((a == e, b < e), c == d)!
\begin{giftFragmento}
El tipo mas general es\end  {giftFragmento}


\begin{giftMO}
\item \begin{giftFragmento}
\end  {giftFragmento}

\verb!foo :: (Ord a, Eq b) \=> a -> a -> b -> b -> a -> Bool!
\begin{giftFragmento}
\end  {giftFragmento}


\item \begin{giftFragmento}
\end  {giftFragmento}

\verb!foo :: (Ord a, Ord b, Eq c, Eq d, Ord e) \=> a -> b -> c -> d -> e -> Bool!
\begin{giftFragmento}
\end  {giftFragmento}


\item \begin{giftFragmento}
\end  {giftFragmento}

\verb!foo :: (Ord a) \=> a -> a -> a -> a -> a -> Bool!
\begin{giftFragmento}
\end  {giftFragmento}


\item \begin{giftFragmento}
No tiene  # Analice el tipo de\end  {giftFragmento}

\verb!((a \=\= e, b < e), c \=\= d)!
\begin{giftFragmento}
. Cuando puedo comparar dos valores?\end  {giftFragmento}


\end  {giftMO}

\begin{giftFragmento}
\end  {giftFragmento}


\begin{giftComentario}
 
\end  {giftComentario}

\begin{giftComentario}
 
\end  {giftComentario}
\giftNombre{Otras_P18_3}
\begin{giftFragmento}
Dada la siguiente definicion:\end  {giftFragmento}

\verb!map' f = (map (uncurry ($))) . box!
\begin{giftFragmento}
¿Cual de las siguientes implementaciones de\end  {giftFragmento}

\verb!box!
\begin{giftFragmento}
hace que\end  {giftFragmento}

\verb!map'!
\begin{giftFragmento}
sea equivalente a\end  {giftFragmento}

\verb!map!
\begin{giftFragmento}
?\end  {giftFragmento}


\begin{giftShort}
\item \begin{giftFragmento}
\end  {giftFragmento}

\verb!zip (repeat f)!
\begin{giftFragmento}
\end  {giftFragmento}


\item \begin{giftFragmento}
No hay implementacion posible de\end  {giftFragmento}

\verb!box!
\begin{giftFragmento}
para eso.\end  {giftFragmento}


\item \begin{giftFragmento}
\end  {giftFragmento}

\verb!zipWith f!
\begin{giftFragmento}
\end  {giftFragmento}


\item \begin{giftFragmento}
\end  {giftFragmento}

\verb!zip f!
\begin{giftFragmento}
\end  {giftFragmento}


\end  {giftShort}

\begin{giftFragmento}
\end  {giftFragmento}


\begin{giftComentario}
 
\end  {giftComentario}

\begin{giftComentario}
 
\end  {giftComentario}
\giftNombre{Otras_P18_4}
\begin{giftFragmento}
Dada la siguiente definicion:\end  {giftFragmento}

\verb!bar f n xs = concat $ map (f n) xs!
\begin{giftFragmento}
Cual de las siguientes afirmaciones no es correcta?\end  {giftFragmento}


\begin{giftMO}
\item \begin{giftFragmento}
\end  {giftFragmento}

\verb!sum . bar const 1!
\begin{giftFragmento}
\end  {giftFragmento}

$\equiv$
\begin{giftFragmento}
\end  {giftFragmento}

\verb!length!
\begin{giftFragmento}
\end  {giftFragmento}


\item \begin{giftFragmento}
El tipo es\end  {giftFragmento}

\verb!bar :: (a -> b -> [c]) -> a -> [b] -> [c]!
\begin{giftFragmento}
\end  {giftFragmento}


\item \begin{giftFragmento}
\end  {giftFragmento}

\verb!bar take 2 [[1,2], [3,4,5]]!
\begin{giftFragmento}
retorna\end  {giftFragmento}

\verb![1,2,3,4]!
\begin{giftFragmento}
\end  {giftFragmento}


\item \begin{giftFragmento}
\end  {giftFragmento}

\verb!bar (\\f x -> [f x])!
\begin{giftFragmento}
\end  {giftFragmento}

$\equiv$
\begin{giftFragmento}
\end  {giftFragmento}

\verb!map!
\begin{giftFragmento}
\end  {giftFragmento}


\end  {giftMO}

\begin{giftFragmento}
\end  {giftFragmento}


\begin{giftComentario}
 
\end  {giftComentario}

\begin{giftComentario}
 
\end  {giftComentario}
\giftNombre{Algebraicos_P18_5}
\begin{giftFragmento}
Cual de las siguientes afirmaciones es correcta?\end  {giftFragmento}


\begin{giftMO}
\item \begin{giftFragmento}
Dado\end  {giftFragmento}

\verb!t!
\begin{giftFragmento}
de tipo\end  {giftFragmento}

\verb!A!
\begin{giftFragmento}
,\end  {giftFragmento}

\verb!snd (copia t)!
\begin{giftFragmento}
siempre diverge o retorna un entero no negativo\end  {giftFragmento}


\item \begin{giftFragmento}
El codigo no compila porque la definicion de\end  {giftFragmento}

\verb!A!
\begin{giftFragmento}
es incorrecta\end  {giftFragmento}


\item \begin{giftFragmento}
El codigo no compila porque la definicion de\end  {giftFragmento}

\verb!A!
\begin{giftFragmento}
es incorrecta\end  {giftFragmento}


\item \begin{giftFragmento}
El codigo no compila porque la definicion de\end  {giftFragmento}

\verb!copia!
\begin{giftFragmento}
es incorrecta\end  {giftFragmento}


\item \begin{giftFragmento}
Dado\end  {giftFragmento}

\verb!t!
\begin{giftFragmento}
de tipo\end  {giftFragmento}

\verb!A!
\begin{giftFragmento}
,\end  {giftFragmento}

\verb!fst (copia t)!
\begin{giftFragmento}
siempre diverge o retorna\end  {giftFragmento}

\verb!t!
\begin{giftFragmento}
\end  {giftFragmento}


\end  {giftMO}

\begin{giftFragmento}
\end  {giftFragmento}


\begin{giftComentario}
 
 
\end  {giftComentario}

\begin{giftComentario}
 
\end  {giftComentario}
\giftNombre{Algebraicos_P18_5}
\begin{giftFragmento}
Cual de las siguientes afirmaciones es correcta?\end  {giftFragmento}


\begin{giftMO}
\item \begin{giftFragmento}
Dado\end  {giftFragmento}

\verb!t!
\begin{giftFragmento}
de tipo\end  {giftFragmento}

\verb!A!
\begin{giftFragmento}
,\end  {giftFragmento}

\verb!snd (copia t)!
\begin{giftFragmento}
siempre diverge o retorna un entero no negativo\end  {giftFragmento}


\item \begin{giftFragmento}
El codigo no compila porque la definicion de\end  {giftFragmento}

\verb!A!
\begin{giftFragmento}
es incorrecta\end  {giftFragmento}


\item \begin{giftFragmento}
El codigo no compila porque la definicion de\end  {giftFragmento}

\verb!copia!
\begin{giftFragmento}
es incorrecta\end  {giftFragmento}


\item \begin{giftFragmento}
Dado\end  {giftFragmento}

\verb!t!
\begin{giftFragmento}
de tipo\end  {giftFragmento}

\verb!A!
\begin{giftFragmento}
,\end  {giftFragmento}

\verb!fst (copia t)!
\begin{giftFragmento}
siempre diverge o retorna\end  {giftFragmento}

\verb!t!
\begin{giftFragmento}
\end  {giftFragmento}


\item \begin{giftFragmento}
El codigo no compila porque la definicion de\end  {giftFragmento}

\verb!A!
\begin{giftFragmento}
es incorrecta\end  {giftFragmento}


\end  {giftMO}

\begin{giftFragmento}
\end  {giftFragmento}


\begin{giftComentario}
 
\end  {giftComentario}

\begin{giftComentario}
 
\end  {giftComentario}
\giftNombre{Algebraicos_P18_6}
\begin{giftFragmento}
Dadas las siguientes definiciones:\end  {giftFragmento}

\verb!data T = C1 | C2 Int Int | C3 T T!
\begin{giftFragmento}
,\end  {giftFragmento}

\verb!baz C1       = 0!
\begin{giftFragmento}
,\end  {giftFragmento}

\verb!baz (C2 x y) = x!
\begin{giftFragmento}
,\end  {giftFragmento}

\verb!baz (C3 x y) = baz x!
\begin{giftFragmento}
.
¿Cual de las siguientes afirmaciones es correcta?\end  {giftFragmento}


\begin{giftMO}
\item \begin{giftFragmento}
El resultado de evaluar\end  {giftFragmento}

\verb!baz (C3 (C2 3 4) (C3 (C2 1 2) C1))!
\begin{giftFragmento}
es\end  {giftFragmento}

\verb!3!
\begin{giftFragmento}
\end  {giftFragmento}


\item \begin{giftFragmento}
El resultado   de evaluar\end  {giftFragmento}

\verb!baz (C3 (C2 3 4) (C3 (C2 1 2) C1))!
\begin{giftFragmento}
es\end  {giftFragmento}

\verb!7!
\begin{giftFragmento}
\end  {giftFragmento}


\item \begin{giftFragmento}
El   resultado de   evaluar\end  {giftFragmento}

\verb!baz (C3 (C2 3 4) (C3 (C2 1 2) C1))!
\begin{giftFragmento}
es\end  {giftFragmento}

\verb!7!
\begin{giftFragmento}
\end  {giftFragmento}


\item \begin{giftFragmento}
El resultado de evaluar\end  {giftFragmento}

\verb!baz (C3 (C2 3 4) (C3 (C2 1 2) C1))!
\begin{giftFragmento}
es\end  {giftFragmento}

\verb!1!
\begin{giftFragmento}
\end  {giftFragmento}


\item \begin{giftFragmento}
El codigo no compila\end  {giftFragmento}


\end  {giftMO}

\begin{giftFragmento}
\end  {giftFragmento}


\begin{giftComentario}
 
\end  {giftComentario}

\begin{giftComentario}
 
\end  {giftComentario}
\giftNombre{Fold_P18_7}
\begin{giftFragmento}
Dada la siguiente definicion:\end  {giftFragmento}

\verb!qux f []     ys = ys!
\begin{giftFragmento}
\end  {giftFragmento}

\verb!qux f (x:xs) ys = qux f xs (f x : ys)!
\begin{giftFragmento}
Cual de las siguientes afirmaciones es correcta?\end  {giftFragmento}


\begin{giftMO}
\item \begin{giftFragmento}
\end  {giftFragmento}

\verb!qux f!
\begin{giftFragmento}
\end  {giftFragmento}

$\equiv$
\begin{giftFragmento}
\end  {giftFragmento}

\verb!flip $ foldl (\\r x -> f x : r)!
\begin{giftFragmento}
\end  {giftFragmento}


\item \begin{giftFragmento}
\end  {giftFragmento}

\verb!qux f!
\begin{giftFragmento}
\end  {giftFragmento}

$\equiv$
\begin{giftFragmento}
\end  {giftFragmento}

\verb!foldl (\\r x -> f x : r)!
\begin{giftFragmento}
\end  {giftFragmento}


\item \begin{giftFragmento}
\end  {giftFragmento}

\verb!qux f!
\begin{giftFragmento}
\end  {giftFragmento}

$\equiv + 2$
\begin{giftFragmento}
\end  {giftFragmento}

\verb!foldl (\\r x -> f x : r)!
\begin{giftFragmento}
\end  {giftFragmento}


\item \begin{giftFragmento}
\end  {giftFragmento}

\verb!qux f!
\begin{giftFragmento}
\end  {giftFragmento}

$\equiv +  2$
\begin{giftFragmento}
\end  {giftFragmento}

\verb!flip $ foldr (\\x r -> f x : r)!
\begin{giftFragmento}
\end  {giftFragmento}


\item \begin{giftFragmento}
\end  {giftFragmento}

\verb!qux f!
\begin{giftFragmento}
\end  {giftFragmento}

$\equiv$
\begin{giftFragmento}
\end  {giftFragmento}

\verb!foldr (\\x r -> f x : r)!
\begin{giftFragmento}
\end  {giftFragmento}


\end  {giftMO}

\begin{giftFragmento}
\end  {giftFragmento}


\begin{giftComentario}
 
\end  {giftComentario}

\end{document}
